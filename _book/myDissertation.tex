% This is the Reed College LaTeX thesis template. Most of the work
% for the document class was done by Sam Noble (SN), as well as this
% template. Later comments etc. by Ben Salzberg (BTS). Additional
% restructuring and APA support by Jess Youngberg (JY).
% Your comments and suggestions are more than welcome; please email
% them to cus@reed.edu
%
% See http://web.reed.edu/cis/help/latex.html for help. There are a
% great bunch of help pages there, with notes on
% getting started, bibtex, etc. Go there and read it if you're not
% already familiar with LaTeX.
%
% Any line that starts with a percent symbol is a comment.
% They won't show up in the document, and are useful for notes
% to yourself and explaining commands.
% Commenting also removes a line from the document;
% very handy for troubleshooting problems. -BTS

% As far as I know, this follows the requirements laid out in
% the 2002-2003 Senior Handbook. Ask a librarian to check the
% document before binding. -SN

%%
%% Preamble
%%
% \documentclass{<something>} must begin each LaTeX document
\documentclass[12pt,twoside]{reedthesis}
% Packages are extensions to the basic LaTeX functions. Whatever you
% want to typeset, there is probably a package out there for it.
% Chemistry (chemtex), screenplays, you name it.
% Check out CTAN to see: http://www.ctan.org/
%%
\usepackage{graphicx,latexsym}
\usepackage{amsmath}
\usepackage{amssymb,amsthm}
\usepackage{longtable,booktabs,setspace}
\usepackage{chemarr} %% Useful for one reaction arrow, useless if you're not a chem major
\usepackage[hyphens]{url}
% Added by CII
\usepackage{hyperref}
\usepackage{lmodern}
\usepackage{float}
\floatplacement{figure}{H}
% End of CII addition
\usepackage{rotating}

% Next line commented out by CII
%%% \usepackage{natbib}
% Comment out the natbib line above and uncomment the following two lines to use the new
% biblatex-chicago style, for Chicago A. Also make some changes at the end where the
% bibliography is included.
%\usepackage{biblatex-chicago}
%\bibliography{thesis}


% Added by CII (Thanks, Hadley!)
% Use ref for internal links
\renewcommand{\hyperref}[2][???]{\autoref{#1}}
\def\chapterautorefname{Chapter}
\def\sectionautorefname{Section}
\def\subsectionautorefname{Subsection}
% End of CII addition

% Added by CII
\usepackage{caption}
\captionsetup{width=5in}
% End of CII addition

% \usepackage{times} % other fonts are available like times, bookman, charter, palatino

% Syntax highlighting #22

% To pass between YAML and LaTeX the dollar signs are added by CII
\title{A dissertation}
\author{Jessica L. Burnett}
% The month and year that you submit your FINAL draft TO THE LIBRARY (May or December)
\date{2019}
\division{}
\advisor{Craig R. Allen}
\institution{University of Nebraska-Lincoln}
\degree{Doctor of Philosophy}
%If you have two advisors for some reason, you can use the following
% Uncommented out by CII
\altadvisor{Dirac Twidwell}
% End of CII addition

%%% Remember to use the correct department!
\department{School of Natural Resources}
% if you're writing a thesis in an interdisciplinary major,
% uncomment the line below and change the text as appropriate.
% check the Senior Handbook if unsure.
%\thedivisionof{The Established Interdisciplinary Committee for}
% if you want the approval page to say "Approved for the Committee",
% uncomment the next line
%\approvedforthe{Committee}

% Added by CII
%%% Copied from knitr
%% maxwidth is the original width if it's less than linewidth
%% otherwise use linewidth (to make sure the graphics do not exceed the margin)
\makeatletter
\def\maxwidth{ %
  \ifdim\Gin@nat@width>\linewidth
    \linewidth
  \else
    \Gin@nat@width
  \fi
}
\makeatother

\renewcommand{\contentsname}{Table of Contents}
% End of CII addition

\setlength{\parskip}{0pt}

% Added by CII

\providecommand{\tightlist}{%
  \setlength{\itemsep}{0pt}\setlength{\parskip}{0pt}}

\Acknowledgements{
Graduate school itself isn't hard, but the journey is. I have a lot of
people and institutions to thank for their emotional, intellectual,
financial, and other professional support. I wish to first highlight how
\textbf{great it was to be a graduate student at this university and in
the School of Natural Resources}. I have received tremendous support at
all levels of the university. Although I am not a fan of Nebraska's
climate, I highly recommend this school to prospective students. First,
I thank my supervisors, Craig Allen and Dirac Twidwell, for providing me
with this amazing opportunity and for supporting my growth as an
independent researcher. I thank my committee members, Craig Allen, David
Angeler, John De Long, Dirac Twidwell, and Drew Tyre for their support
and advisement, but especially for their comprehensive examination--I
found this process transformative. I especially thank Dirac for his
comprehensive exam questions--I never knew how much theory I didn't know
until I studied your list\ldots{} \textbf{Financial support}. This
research was funded by the U.S. Department of Defense's Strategic
Environmental Research and Development Program (project ID: RC-2510).
The University of Nebraska-Lincoln (UNL) has been highy supportive in my
doctoral studies and reserach. I am grateful for the generous of donors
to the University of Nebraska Foundation, which provided me with two
prestigious supplemental fellowships: Fling and Othmer. I also thank the
Nelson Family (Nelson Memorial Fellowship) and the Institute of
Agriculture and Natural Resources, who funded large portions of my
academic and research-related travel. I thank the School of Natural
Resources for their financial support in my conference travel. The U.S.
National Academy of Sciences generously funded part of my travel to the
International Institute for Applied Systems Analysis (IIASA). This
financial support provided me not only with invaluabe opportunities to
attend and present at national and international conferences and
workshops, conduct research abroad, and network--this funding alleviated
some financial pressures associated with graduate school which allowed a
more refined focus on my research. The opportunities and experiences
provided to me by these funding sources were amazing, thank you all.
\textbf{Emotional support}. I am one of the many graduate students
afflicted with mental health ``disorders'' which negatively impact my
quality of life, at times. I am first grafteful to one friend who
unknowingly destigmatized mental health, without which I may not have
sought treatment and diagnosis--thank you, Hannah. Since my diagnoses, I
have tried to encourage this destigmatization among graduate students in
our department. I thank felow students and faculty who have also been
outspoken regarding related issues (Jamilynn Polletto and Drew Tyre).
Finally, I thank Terry Thomas for her patience, support, and knowledge
as my general practitioner and mental health advocate.\\
I thank others for their various and probably unknowing contributions to
my professional development: David Angeler, Christie Bahlai, Hannah
Birge, Mary Bomberger Brown, John Carroll, Jenny Dauer, John DeLong,
Tarsha Eason, Brian Fath, Ahjond Garmestani, Chris Lepczyk, Frank La
Sorte, Chai Molina, Erica Stuber, Zac Warren, Lyndsie Wszola, Hao Ye,
Peter Zebrowski. I would like to especially thank some of the amazing
and brilliant \textbf{female scientists} in my life for their
encouragement: Jane Anderson, Hannah Birge, Mary Bomberger Brown, Tori
Donovan, Brittany Dueker, Allie Schiltmeyer, Katie Sieving, Erica
Stuber, and Lyndsie Wszola.
\begin{enumerate}
\def\labelenumi{\arabic{enumi}.}
\item
  \textbf{Federal employment}. One reason for coming to this program was
  specifically to study in a USGS Cooperative Research Unit, and to
  understand better life as a federal scientist. I thank Craig Allen and
  Kevin Pope for entertaining many hours of discussion (interrogation?)
  regarding federal employment.
\item
  \textbf{IIASA}. Studying at the International Institute for Applied
  Systems Analysis was an amazing opportunity. I thank Brian Fath and
  Elena Rovenskaya for their advisement, members of the Applied Systems
  Analysis research group for their feedback on my research, and to the
  postdocs and YSSPers.
\end{enumerate}
HEB, TD, CPR, DF, CRA, BF, ER, CAL, MPM, KES, FL, MBB,
\begin{enumerate}
\def\labelenumi{\arabic{enumi}.}
\tightlist
\item
  \textbf{Professional development}. AJT, KP, CRA, DT, MBB, JC,
\end{enumerate}
To my partner of eight years--Schultzie--thank you for everything. Just
kidding, thank you, Nat Price.
}

\Dedication{
Something snarky to mike moulton -- maybe a limerick
}

\Preface{
This is an example of a thesis setup to use the reed thesis document
class (for LaTeX) and the R bookdown package, in general.
}

\Abstract{
THis is my amazing abstract.
}

% End of CII addition
%%
%% End Preamble
%%
%
\begin{document}

% Everything below added by CII
  \maketitle

\frontmatter % this stuff will be roman-numbered
\pagestyle{empty} % this removes page numbers from the frontmatter
  \begin{acknowledgements}
    Graduate school itself isn't hard, but the journey is. I have a lot of
    people and institutions to thank for their emotional, intellectual,
    financial, and other professional support. I wish to first highlight how
    \textbf{great it was to be a graduate student at this university and in
    the School of Natural Resources}. I have received tremendous support at
    all levels of the university. Although I am not a fan of Nebraska's
    climate, I highly recommend this school to prospective students. First,
    I thank my supervisors, Craig Allen and Dirac Twidwell, for providing me
    with this amazing opportunity and for supporting my growth as an
    independent researcher. I thank my committee members, Craig Allen, David
    Angeler, John De Long, Dirac Twidwell, and Drew Tyre for their support
    and advisement, but especially for their comprehensive examination--I
    found this process transformative. I especially thank Dirac for his
    comprehensive exam questions--I never knew how much theory I didn't know
    until I studied your list\ldots{} \textbf{Financial support}. This
    research was funded by the U.S. Department of Defense's Strategic
    Environmental Research and Development Program (project ID: RC-2510).
    The University of Nebraska-Lincoln (UNL) has been highy supportive in my
    doctoral studies and reserach. I am grateful for the generous of donors
    to the University of Nebraska Foundation, which provided me with two
    prestigious supplemental fellowships: Fling and Othmer. I also thank the
    Nelson Family (Nelson Memorial Fellowship) and the Institute of
    Agriculture and Natural Resources, who funded large portions of my
    academic and research-related travel. I thank the School of Natural
    Resources for their financial support in my conference travel. The U.S.
    National Academy of Sciences generously funded part of my travel to the
    International Institute for Applied Systems Analysis (IIASA). This
    financial support provided me not only with invaluabe opportunities to
    attend and present at national and international conferences and
    workshops, conduct research abroad, and network--this funding alleviated
    some financial pressures associated with graduate school which allowed a
    more refined focus on my research. The opportunities and experiences
    provided to me by these funding sources were amazing, thank you all.
    \textbf{Emotional support}. I am one of the many graduate students
    afflicted with mental health ``disorders'' which negatively impact my
    quality of life, at times. I am first grafteful to one friend who
    unknowingly destigmatized mental health, without which I may not have
    sought treatment and diagnosis--thank you, Hannah. Since my diagnoses, I
    have tried to encourage this destigmatization among graduate students in
    our department. I thank felow students and faculty who have also been
    outspoken regarding related issues (Jamilynn Polletto and Drew Tyre).
    Finally, I thank Terry Thomas for her patience, support, and knowledge
    as my general practitioner and mental health advocate.\\
    I thank others for their various and probably unknowing contributions to
    my professional development: David Angeler, Christie Bahlai, Hannah
    Birge, Mary Bomberger Brown, John Carroll, Jenny Dauer, John DeLong,
    Tarsha Eason, Brian Fath, Ahjond Garmestani, Chris Lepczyk, Frank La
    Sorte, Chai Molina, Erica Stuber, Zac Warren, Lyndsie Wszola, Hao Ye,
    Peter Zebrowski. I would like to especially thank some of the amazing
    and brilliant \textbf{female scientists} in my life for their
    encouragement: Jane Anderson, Hannah Birge, Mary Bomberger Brown, Tori
    Donovan, Brittany Dueker, Allie Schiltmeyer, Katie Sieving, Erica
    Stuber, and Lyndsie Wszola.
    \begin{enumerate}
    \def\labelenumi{\arabic{enumi}.}
    \item
      \textbf{Federal employment}. One reason for coming to this program was
      specifically to study in a USGS Cooperative Research Unit, and to
      understand better life as a federal scientist. I thank Craig Allen and
      Kevin Pope for entertaining many hours of discussion (interrogation?)
      regarding federal employment.
    \item
      \textbf{IIASA}. Studying at the International Institute for Applied
      Systems Analysis was an amazing opportunity. I thank Brian Fath and
      Elena Rovenskaya for their advisement, members of the Applied Systems
      Analysis research group for their feedback on my research, and to the
      postdocs and YSSPers.
    \end{enumerate}
    HEB, TD, CPR, DF, CRA, BF, ER, CAL, MPM, KES, FL, MBB,
    \begin{enumerate}
    \def\labelenumi{\arabic{enumi}.}
    \tightlist
    \item
      \textbf{Professional development}. AJT, KP, CRA, DT, MBB, JC,
    \end{enumerate}
    To my partner of eight years--Schultzie--thank you for everything. Just
    kidding, thank you, Nat Price.
  \end{acknowledgements}
  \begin{preface}
    This is an example of a thesis setup to use the reed thesis document
    class (for LaTeX) and the R bookdown package, in general.
  \end{preface}
  \hypersetup{linkcolor=black}
  \setcounter{tocdepth}{2}
  \tableofcontents

  \listoftables

  \listoffigures
  \begin{abstract}
    THis is my amazing abstract.
  \end{abstract}
  \begin{dedication}
    Something snarky to mike moulton -- maybe a limerick
  \end{dedication}
\mainmatter % here the regular arabic numbering starts
\pagestyle{fancyplain} % turns page numbering back on

\chapter{thesisdown::thesis\_word:
default}\label{thesisdownthesis_word-default}

Placeholder

\chapter*{Preliminary Content}\label{preliminary-content}
\addcontentsline{toc}{chapter}{Preliminary Content}

Paste from index.rmd if knitting to git or html

\section*{Acknowledgements}\label{acknowledgements}
\addcontentsline{toc}{section}{Acknowledgements}

Paste from index.rmd if knitting to git or html

\section*{Preface}\label{preface}
\addcontentsline{toc}{section}{Preface}

Paste from index.rmd if knitting to git or html

\section*{Dedication}\label{dedication}
\addcontentsline{toc}{section}{Dedication}

Paste from index.rmd if knitting to git or html

THis is my amazing abstract.

\chapter{Introduction}\label{intro-chapter}

\section{Background}\label{background}
\begin{itemize}
\tightlist
\item
  On abrupt changes in the environment
\end{itemize}
\begin{enumerate}
\def\labelenumi{\arabic{enumi}.}
\tightlist
\item
  A few examples of abrupt changes that are highly referenced.\\
\item
  Whydoes it matter that we can detect??
\item
  A few examples of the methds that have been used ot identifythese
  shifts
\end{enumerate}
\begin{itemize}
\tightlist
\item
  histortically
\item
  real-time
\item
  predictive
\end{itemize}
\begin{enumerate}
\def\labelenumi{\arabic{enumi}.}
\setcounter{enumi}{2}
\tightlist
\item
  PRoblems with the methods in
\end{enumerate}
\begin{itemize}
\tightlist
\item
  aplpication
  \begin{itemize}
  \tightlist
  \item
    difficult to apply
  \item
    to interpret
  \end{itemize}
\item
  theory - lackthereof
\end{itemize}
\begin{enumerate}
\def\labelenumi{\arabic{enumi}.}
\setcounter{enumi}{3}
\tightlist
\item
  Descrive the attempts to identify regime shifts
\end{enumerate}
\section{My thesis}\label{my-thesis}

My thesis is that regime detection metrics are not useful and are
difficult to interpret and apply to multispecies systems. 1.
Brandolini's principle

TWo major sources of problems? 1. Defining a regime shift\\
2. Methods have not proven useful for application beyond single-species
systems and systems about which causal drivers can just be monitored.
\begin{itemize}
\tightlist
\item
  Current state of regime shift theory
\item
  Why it is important to diagnose/detect abrupt changes at the system
  level
\item
  Current methods are not being employed by ecological management.
  \begin{itemize}
  \tightlist
  \item
    Why are applications largely restricted to theoretical research?
  \item
    Why are the applications to empirical data largely restricted to the
    research community?
  \item
    Is this an artefact of how long it takes for applied ecologists and
    ecological management to adopt new data anlysis techniques?
  \end{itemize}
\end{itemize}
\section{Dissertation abstract (content
summary)}\label{dissertation-abstract-content-summary}

This dissertation comprises \textbf{X} sections:\\
1. Review of the current methods used to identify abrupt changes in
ecological systems - Types of analyses - Univariable vs.~multivariable -
Picked up vs.~not picked up (look at \# papers using method in WOS,
maybe\ldots{})\\
1. A beginner's guide to Fisher Information (derivatives metric)
\{\#distance\}\\
1. Distance method\\
1. Fisher Information binning method and an application of it to
spatiotemporal data 1. Conclusions

\chapter{Quantitative indicators of abrupt ecological
change}\label{indicators-chapter}

Placeholder

\section{Abstract}\label{abstract}

\section{Introduction}\label{introduction}

\section{Methods}\label{methods}

\subsection{Identifying papers/RSDMs in the
literature}\label{identifying-papersrsdms-in-the-literature}

\section{Results}\label{results}

\subsection{Potential figures}\label{potential-figures}

\subsection{Potential tables}\label{potential-tables}

\section{Discussion}\label{discussion}

\chapter{A guide to Fisher Information for Ecologists}\label{fiGuide}

Placeholder

\section{Abstract}\label{abstract-1}

\section{Introduction}\label{introduction-1}

\subsubsection{\texorpdfstring{\textbf{Step 3.} \(p(s)\) as a function
of the rate of change of
\(s\)}{Step 3. p(s) as a function of the rate of change of s}}\label{step-3.-ps-as-a-function-of-the-rate-of-change-of-s}

\subsubsection{\texorpdfstring{\textbf{Step 4.} Calculate the
derivatives-based Fisher
Information}{Step 4. Calculate the derivatives-based Fisher Information}}\label{step-4.-calculate-the-derivatives-based-fisher-information}

\section{Acknowledgements}\label{acknowledgements-1}

\chapter{An application of the Fisher Information binning method to
spatiotemporal avian community
data}\label{an-application-of-the-fisher-information-binning-method-to-spatiotemporal-avian-community-data}

Placeholder

\section{Abstract}\label{abstract-2}

\section{Introduction}\label{introduction-2}

\section{Methods}\label{methods-1}

\subsection{Data collection}\label{data-collection}

\subsection{Study areas}\label{study-areas}

\subsubsection{Military bases as study
sites}\label{military-bases-as-study-sites}

\subsubsection{Focal military bases}\label{focal-military-bases}

\subsubsection{Delineating spatial transects for spatial
analysis}\label{delineating-spatial-transects-for-spatial-analysis}

\subsubsection{Delineating transects for spatial
analysis}\label{delineating-transects-for-spatial-analysis}

\subsubsection{Selecting routes for temporal
analysis}\label{selecting-routes-for-temporal-analysis}

\subsection{Calculating the Fisher Information binning
measure}\label{calculating-the-fisher-information-binning-measure}

\section{Results}\label{results-1}

\subsection{Temporal data}\label{temporal-data}

\subsection{Spatial data}\label{spatial-data}

\subsection{Interpreting the Fisher Information binning
measure}\label{interpreting-the-fisher-information-binning-measure}

\section{Discussion}\label{discussion-1}

\appendix

\chapter{The First Appendix}\label{the-first-appendix}

This first appendix includes all of the R chunks of code that were
hidden throughout the document (using the \texttt{include\ =\ FALSE}
chunk tag) to help with readibility and/or setup.

\textbf{In the main Rmd file}

\textbf{In Chapter \ref{ref-labels}:}

\chapter{The Second Appendix, for
Fun}\label{the-second-appendix-for-fun}

\chapter*{References}\label{references}
\addcontentsline{toc}{chapter}{References}

Placeholder


% Index?

\end{document}
