% This is the Reed College LaTeX thesis template. Most of the work
% for the document class was done by Sam Noble (SN), as well as this
% template. Later comments etc. by Ben Salzberg (BTS). Additional
% restructuring and APA support by Jess Youngberg (JY).
% Your comments and suggestions are more than welcome; please email
% them to cus@reed.edu
%
% See http://web.reed.edu/cis/help/latex.html for help. There are a
% great bunch of help pages there, with notes on
% getting started, bibtex, etc. Go there and read it if you're not
% already familiar with LaTeX.
%
% Any line that starts with a percent symbol is a comment.
% They won't show up in the document, and are useful for notes
% to yourself and explaining commands.
% Commenting also removes a line from the document;
% very handy for troubleshooting problems. -BTS

% As far as I know, this follows the requirements laid out in
% the 2002-2003 Senior Handbook. Ask a librarian to check the
% document before binding. -SN

%%
%% Preamble
%%
% \documentclass{<something>} must begin each LaTeX document
\documentclass[12pt,twoside,openany]{reedthesis}
% Packages are extensions to the basic LaTeX functions. Whatever you
% want to typeset, there is probably a package out there for it.
% Chemistry (chemtex), screenplays, you name it.
% Check out CTAN to see: http://www.ctan.org/
%%

\usepackage{setspace}
\usepackage{graphicx,latexsym}
\usepackage{amsmath}
\usepackage{amssymb,amsthm}
%\usepackage{longtable,booktabs,setspace}
\usepackage{longtable,booktabs,setspace, array}
\usepackage{chemarr} %% Useful for one reaction arrow, useless if you're not a chem major
\usepackage[hyphens]{url}
% Added by CII
\usepackage{hyperref}
\usepackage{lmodern}
\usepackage{float}
\floatplacement{figure}{H}
% End of CII addition
\usepackage{rotating}
% Next line commented out by CII
%%% \usepackage{natbib}
% Comment out the natbib line above and uncomment the following two lines to use the new
% biblatex-chicago style, for Chicago A. Also make some changes at the end where the
% bibliography is included.
%\usepackage{biblatex-chicago}
%\bibliography{thesis}


% Added by CII (Thanks, Hadley!)
% Use ref for internal links
\renewcommand{\hyperref}[2][???]{\autoref{#1}}
\def\chapterautorefname{Chapter}
\def\sectionautorefname{Section}
\def\subsectionautorefname{Subsection}
% End of CII addition

% Added by CII
\usepackage{caption}
\captionsetup{width=5in}
% End of CII addition

% \usepackage{times} % other fonts are available like times, bookman, charter, palatino

% Syntax highlighting #22

% To pass between YAML and LaTeX the dollar signs are added by CII
\title{Regime Detection Measures for the Practical Ecologist}
\author{Jessica L. Burnett}
% The month and year that you submit your FINAL draft TO THE LIBRARY
\date{2019}
% \division{}
\advisor{Craig R. Allen}
\department{School of Natural Resources}
\institution{University of Nebraska-Lincoln}
\degree{Doctor of Philosophy}
%If you have two advisors for some reason, you can use the following
% Uncommented out by CII
\altadvisor{Dirac Twidwell}
% End of CII addition

%%% Remember to use the correct department!
% if you're writing a thesis in an interdisciplinary major,
% uncomment the line below and change the text as appropriate.
% check the Senior Handbook if unsure.
%\thedivisionof{The Established Interdisciplinary Committee for}
% if you want the approval page to say "Approved for the Committee",
% uncomment the next line
%\approvedforthe{Committee}

% Added by CII
%%% Copied from knitr
%% maxwidth is the original width if it's less than linewidth
%% otherwise use linewidth (to make sure the graphics do not exceed the margin)
\makeatletter
\def\maxwidth{ %
  \ifdim\Gin@nat@width>\linewidth
    \linewidth
  \else
    \Gin@nat@width
  \fi
}
\makeatother

\renewcommand{\contentsname}{Table of Contents}
% End of CII addition

\setlength{\parskip}{0pt}

% Added by CII

\providecommand{\tightlist}{%
  \setlength{\itemsep}{0pt}\setlength{\parskip}{0pt}}

\Acknowledgements{
Graduate school itself isn't hard, but the journey is. I have a lot of
people and institutions to thank for their emotional, intellectual,
financial, and other support. I wish to first highlight how
\textbf{great it was to be a graduate student at this university and in
the School of Natural Resources}. UNL has provided tremendous support at
all levels of the university. Although I am not a fan of Nebraska's
climate, I highly recommend this school to prospective students. I thank
my supervisors, Craig Allen and Dirac Twidwell, for providing me with
this amazing opportunity and for supporting my growth as an independent
researcherm and my committee members, Craig Allen, David Angeler, John
De Long, Dirac Twidwell, and Drew Tyre for their support and advisement,
but especially for their comprehensive examination--I found this process
transformative, albeit very stress inducing. I also wish to thank Dirac
for his comprehensive exam questions--I never knew how much I didn't
know until I studied your recommendations. I also thank Craig for
supporting my efforts to study and conduct research outside of our
immediate geographical settings. Studying at the International Institute
for Applied Systems Analysis was an amazing opportunity! I thank Brian
Fath and Elena Rovenskaya for their advisement, members of the Applied
Systems Analysis research group for their feedback on my research, and
to the postdocs and YSSPers. I would like to especially thank some of
the amazing and brilliant \textbf{female scientists} in my life for
their encouragement: Jane Anderson, Karen Bailey, Hannah Birge, Mary
Bomberger Brown, Tori Donovan, Brittany Dueker, Allie Schiltmeyer, Katie
Sieving, Erica Stuber, Becky Wilcox, Carissa Wonkka, and Lyndsie Wszola.
I thank these women and others for their contributions to my
professional development: David Angeler, Christie Bahlai, Mary Bomberger
Brown, John Carroll, Jenny Dauer, John DeLong, Tarsha Eason, Brian Fath,
Ahjond Garmestani, Chris Lepczyk, Frank La Sorte, Chai Molina, Zac
Warren, Hao Ye and Peter Zebrowski. I owe thanks to Craig Allen and
Kevin Pope for entertaining my many hours of discussion (interrogation?)
regarding federal employment. I also thank fellow graduate students whom
I hope I have forged strong and lasting connections: Hannah Birge, Tori
Donovan, Caleb Roberts, Allie Schiltmeyer, and Lyndsie Wszola I am one
of the many graduate students afflicted with mental health
``disorders''. I am first grafteful to one friend (H) who unknowingly
destigmatized mental health in my mind and wihtout whom I may not have
sought treatment. I applaud students and faculty who have been outspoken
regarding mental health related issues, and I am indebtted to my general
practitioner and mental health advocate, Terry Thomas M.A., M.S.N.,
A.P.R.N.\\
\emph{Financial support}. This research was funded by the U.S.
Department of Defense's Strategic Environmental Research and Development
Program (project ID: RC-2510). The University of Nebraska-Lincoln (UNL)
has been highy supportive in my doctoral studies and reserach. I am
grateful for the generous of donors to the University of Nebraska
Foundation, which provided me with two prestigious supplemental
fellowships: Fling and Othmer. I also thank the Nelson Family (Nelson
Memorial Fellowship) and the Institute of Agriculture and Natural
Resources, who funded large portions of my academic and research-related
travel. I thank the School of Natural Resources for their financial
support in my conference travel. The U.S. National Academy of Sciences
generously funded part of my travel to the International Institute for
Applied Systems Analysis (IIASA). This financial support provided me not
only with invaluabe opportunities to attend and present at national and
international conferences and workshops, conduct research abroad, and
network--this funding alleviated some financial pressures associated
with graduate school which allowed a more refined focus on my
dissertation research. The opportunities and experiences provided to me
by each funding source were amazing, thank you. Finally, to my partner
of eight years--Schultzie--thank you for everything. Just kidding, thank
you, Nat Price, you are amazing.
}

\Dedication{
To the end-users and researchers frustrated with jargon and lack of
practical utility of ecological models and metrics. And to Mike Moulton,
without whose support and encouragement many years ago two advanced
degrees would likely not have been possible.
}

\Preface{

}

\Abstract{

}

% End of CII addition
%%
%% End Preamble
%%
%
\begin{document}

% Everything below added by CII
  \maketitle

\frontmatter % this stuff will be roman-numbered
\pagestyle{empty} % this removes page numbers from the frontmatter
  \begin{acknowledgements}
    Graduate school itself isn't hard, but the journey is. I have a lot of
    people and institutions to thank for their emotional, intellectual,
    financial, and other support. I wish to first highlight how
    \textbf{great it was to be a graduate student at this university and in
    the School of Natural Resources}. UNL has provided tremendous support at
    all levels of the university. Although I am not a fan of Nebraska's
    climate, I highly recommend this school to prospective students. I thank
    my supervisors, Craig Allen and Dirac Twidwell, for providing me with
    this amazing opportunity and for supporting my growth as an independent
    researcherm and my committee members, Craig Allen, David Angeler, John
    De Long, Dirac Twidwell, and Drew Tyre for their support and advisement,
    but especially for their comprehensive examination--I found this process
    transformative, albeit very stress inducing. I also wish to thank Dirac
    for his comprehensive exam questions--I never knew how much I didn't
    know until I studied your recommendations. I also thank Craig for
    supporting my efforts to study and conduct research outside of our
    immediate geographical settings. Studying at the International Institute
    for Applied Systems Analysis was an amazing opportunity! I thank Brian
    Fath and Elena Rovenskaya for their advisement, members of the Applied
    Systems Analysis research group for their feedback on my research, and
    to the postdocs and YSSPers. I would like to especially thank some of
    the amazing and brilliant \textbf{female scientists} in my life for
    their encouragement: Jane Anderson, Karen Bailey, Hannah Birge, Mary
    Bomberger Brown, Tori Donovan, Brittany Dueker, Allie Schiltmeyer, Katie
    Sieving, Erica Stuber, Becky Wilcox, Carissa Wonkka, and Lyndsie Wszola.
    I thank these women and others for their contributions to my
    professional development: David Angeler, Christie Bahlai, Mary Bomberger
    Brown, John Carroll, Jenny Dauer, John DeLong, Tarsha Eason, Brian Fath,
    Ahjond Garmestani, Chris Lepczyk, Frank La Sorte, Chai Molina, Zac
    Warren, Hao Ye and Peter Zebrowski. I owe thanks to Craig Allen and
    Kevin Pope for entertaining my many hours of discussion (interrogation?)
    regarding federal employment. I also thank fellow graduate students whom
    I hope I have forged strong and lasting connections: Hannah Birge, Tori
    Donovan, Caleb Roberts, Allie Schiltmeyer, and Lyndsie Wszola I am one
    of the many graduate students afflicted with mental health
    ``disorders''. I am first grafteful to one friend (H) who unknowingly
    destigmatized mental health in my mind and wihtout whom I may not have
    sought treatment. I applaud students and faculty who have been outspoken
    regarding mental health related issues, and I am indebtted to my general
    practitioner and mental health advocate, Terry Thomas M.A., M.S.N.,
    A.P.R.N.\\
    \emph{Financial support}. This research was funded by the U.S.
    Department of Defense's Strategic Environmental Research and Development
    Program (project ID: RC-2510). The University of Nebraska-Lincoln (UNL)
    has been highy supportive in my doctoral studies and reserach. I am
    grateful for the generous of donors to the University of Nebraska
    Foundation, which provided me with two prestigious supplemental
    fellowships: Fling and Othmer. I also thank the Nelson Family (Nelson
    Memorial Fellowship) and the Institute of Agriculture and Natural
    Resources, who funded large portions of my academic and research-related
    travel. I thank the School of Natural Resources for their financial
    support in my conference travel. The U.S. National Academy of Sciences
    generously funded part of my travel to the International Institute for
    Applied Systems Analysis (IIASA). This financial support provided me not
    only with invaluabe opportunities to attend and present at national and
    international conferences and workshops, conduct research abroad, and
    network--this funding alleviated some financial pressures associated
    with graduate school which allowed a more refined focus on my
    dissertation research. The opportunities and experiences provided to me
    by each funding source were amazing, thank you. Finally, to my partner
    of eight years--Schultzie--thank you for everything. Just kidding, thank
    you, Nat Price, you are amazing.
  \end{acknowledgements}

  \hypersetup{linkcolor=black}
  \setcounter{tocdepth}{2}
  \tableofcontents

  \listoftables

  \listoffigures

  \begin{dedication}
    To the end-users and researchers frustrated with jargon and lack of
    practical utility of ecological models and metrics. And to Mike Moulton,
    without whose support and encouragement many years ago two advanced
    degrees would likely not have been possible.
  \end{dedication}
\mainmatter % here the regular arabic numbering starts
\pagestyle{fancyplain} % turns page numbering back on
% \doublespacing{} % Trying to set spacing between lines in body
\linespread{1} % Trying to set spacing between lines in body

\chapter{thesisdown::thesis\_gitbook:
default}\label{thesisdownthesis_gitbook-default}

Placeholder

Identifying abrupt changes in the structure and functioning of systems,
or system regime shifts, in ecological and social-ecological systems
leads to an understanding of relative and absolute system resilience.
Resilience is an emergent phenomenon of complex social-ecological
systems, and is the ability of a system to absorb disturbance without
reorganizing into a new state, or regime. Resilience science provides a
framework and methodology for quantitatively assessing the capacity of a
system to maintain its current trajectory (or to stay within a certain,
and often desirable regime). If and when a system's resilience is
exceeded, it crosses a threshold and enters into an alternate regime (or
undergoes a regime shift).\\
I will use Fisher Information to detect regime shifts in time and space
using avian community data obtained from the North American Breeding
Bird Survey within the area east of the Rockies and west of the
Mississippi River. Fisher Information is a technique that captures the
dynamic of a system, and this metric will be calculated about a suite of
bird species abundances aggregated to the route level for all possible
time periods. Transmutation (aggregation error) about inclusion or
exclusion of certain bird species, functional groups, and guilds will be
analyzed. Efforts have been made to develop early warning indicators of
regime shifts in ecosystems, however, for most ecosystems there is great
uncertainty in predicting the risk of a regime shift, regarding both
when and how long it will take to happen and if it can be recognized
early enough to be avoided when desired. We will complement the use of
Fisher Information with multiple discontinuity analyses about body mass
distributions at the route-level to achieve the aim of identifying
individual species that best serve as early-warning indicators of regime
shifts. For those species found on the edges of body mass aggregations,
we test the hypothesis that the background variance in their abundances
(on Breeding Bird Survey routes) will increase more than those not
observed at the edge of discontinuity aggregations. Identification of
early-warning indicators of regime shifts in ecological systems allows
management efforts to focus on a single or a small number of species
that inform us about ecosystem resilience and trajectory.\\
These methods transcend the primary objective of the Breeding Bird
Survey (to monitor population trends) and use this expansive dataset in
such a way that information about ecosystem order, trajectory, and
resilience emerge. Here, we utilize an expansive dataset (the Breeding
Bird Survey) to make broad-scale estimations and predictions about
ecosystem resilience, regime status and trajectory, and ecosystem
sustainability. Identification of regime shifts and early-warning
indicator species may afford us the ability to predict system regime
shifts in time.

\chapter{Introduction}\label{intro}

Placeholder

\subsection{Dissertation structure}\label{dissertation-structure}

\section{Glossary}\label{glossary}

\chapter{A brief overview of ecological regime detection methods
methods}\label{rdmReview}

Placeholder

\section{Introduction}\label{introduction}

\section{Methods}\label{methods}

\subsection{Identifying candidate
articles}\label{identifying-candidate-articles}

\subsubsection{Web of Science}\label{web-of-science}

\subsubsection{Prior knowledge and snowball
method}\label{prior-knowledge-and-snowball-method}

\subsubsection{Google Scholar}\label{google-scholar}

\subsubsection{Additional filtering}\label{additional-filtering}

\section{Results}\label{results}

\subsection{Web of Science}\label{web-of-science-1}

\subsection{Google Scholar and prior
knowledge}\label{google-scholar-and-prior-knowledge}

\subsection{List of new methods}\label{list-of-new-methods}

\section{Discussion}\label{discussion}

\subsection{Barriers to identifying new
RDMs}\label{barriers-to-identifying-new-rdms}

\subsection{Reducing the barriers to
RDMs}\label{reducing-the-barriers-to-rdms}

\chapter{A guide to Fisher Information for Ecologists}\label{fiGuide}

Placeholder

\section{Abstract}\label{abstract}

\section{Introduction}\label{introduction-1}

\subsection{On Fisher Information}\label{on-fisher-information}

\subsection{Notation}\label{notation}

\subsection{Steps for calculating Fisher Information
(FI)}\label{steps-for-calculating-fisher-information-fi}

\subsection{Concepts behind the
calculations}\label{concepts-behind-the-calculations}

\subsubsection{\texorpdfstring{\textbf{Step 1. Probability of observing
the system in a particular state,
\(p(x)\)}}{Step 1. Probability of observing the system in a particular state, p(x)}}\label{step-1.-probability-of-observing-the-system-in-a-particular-state-px}

\subsubsection{\texorpdfstring{\textbf{Step 2.} Distance traveled by the
system,
\(s\)}{Step 2. Distance traveled by the system, s}}\label{step-2.-distance-traveled-by-the-system-s}

\subsubsection{\texorpdfstring{\textbf{Step 3.} \(p(s)\) as a function
of the rate of change of
\(s\)}{Step 3. p(s) as a function of the rate of change of s}}\label{step-3.-ps-as-a-function-of-the-rate-of-change-of-s}

\subsubsection{\texorpdfstring{\textbf{Step 4.} Calculate the
derivatives-based Fisher
Information}{Step 4. Calculate the derivatives-based Fisher Information}}\label{step-4.-calculate-the-derivatives-based-fisher-information}

\section{Case Study}\label{case-study}

\section{Conclusions}\label{conclusions}

\section{Acknowledgements}\label{acknowledgements}

\chapter{An application of Fisher Information to spatially-explicit
avian community data}\label{fisherSpatial}

Placeholder

\section{Introduction}\label{introduction-2}

\section{Data and Methods}\label{data-and-methods}

\subsection{Data: North American Breeding Bird
Survey}\label{data-north-american-breeding-bird-survey}

\subsection{Study area}\label{study-area}

\subsubsection{Focal military base}\label{focal-military-base}

\subsubsection{Spatial sampling grid}\label{spatial-sampling-grid}

\subsection{Fisher Information (FI)}\label{fisher-information-fi}

\subsubsection{Interpreting FI}\label{interpreting-fi}

\subsection{Efficacy of Fisher Information as a spatial
RDM}\label{efficacy-of-fisher-information-as-a-spatial-rdm}

\subsubsection{Linear interpolation of Fisher
Information}\label{linear-interpolation-of-fisher-information}

\subsubsection{Correlation of Fisher Information among adjacent spatial
tranects}\label{correlation-of-fisher-information-among-adjacent-spatial-tranects}

\section{Results}\label{results-1}

\subsection{Fisher Information}\label{fisher-information}

\subsection{Efficacy of Fisher Information as a spatial
RDM}\label{efficacy-of-fisher-information-as-a-spatial-rdm-1}

\section{Discussion}\label{discussion-1}

\chapter{Data Quality Impacts on Regime Detection
Measures}\label{resampling}

Placeholder

\section{Introduction}\label{introduction-3}

\section{Methods}\label{methods-1}

\subsection{Study system and data}\label{study-system-and-data}

\subsection{Distance travelled metric}\label{distance-travelled-metric}

\section{Results}\label{results-2}

\section{Discussion}\label{discussion-2}

\section{Ackowledgements}\label{ackowledgements}

\chapter{System velocity: rate-of-change of a system's trajectory in
phase space as an indicator of abrupt change in high-dimensional
data}\label{velocity}

\section{Introduction}\label{introduction-4}

\appendix

\chapter*{Appendix A}\label{rRDM}
\addcontentsline{toc}{chapter}{Appendix A}

This appendix contains the vignette associated with the R Package,
\texttt{rRDM}. Development source code for this package is available on
GitHub as a compressed file,
\url{https://github.com/TrashBirdEcology/rRDM/archive/master.zip} or
\url{https://github.com/TrashBirdEcology/rRDM}.

\chapter*{References}\label{references}
\addcontentsline{toc}{chapter}{References}

Placeholder


\end{document}
